Система диференціальних рівнянь, що записана у вигляді
\begin{equation*}
	\left\{
		\begin{array}{rl}
			\dot x_1 &= a_{11}(t) x_1 + a_{12}(t) x_2 + \ldots + a_{1n}(t) x_n + f_1(t), \\
			\dot x_2 &= a_{21}(t) x_1 + a_{22}(t) x_2 + \ldots + a_{2n}(t) x_n + f_2(t), \\
			\hdotsfor{2} \\
			\dot x_n &= a_{n1}(t) x_1 + a_{n2}(t) x_2 + \ldots + a_{nn}(t) x_n + f_n(t),
		\end{array}
	\right.
\end{equation*}
називається лінійною неоднорідною системою диференціальних рівнянь. Система 
\begin{equation*}
	\left\{
		\begin{array}{rl}
			\dot x_1 &= a_{11}(t) x_1 + a_{12}(t) x_2 + \ldots + a_{1n}(t) x_n, \\
			\dot x_2 &= a_{21}(t) x_1 + a_{22}(t) x_2 + \ldots + a_{2n}(t) x_n, \\
			\hdotsfor{2} \\
			\dot x_n &= a_{n1}(t) x_1 + a_{n2}(t) x_2 + \ldots + a_{nn}(t) x_n,
		\end{array}
	\right.
\end{equation*}
називається лінійною однорідною системою диференціальних рівнянь. Якщо ввести векторні позначення
\begin{equation*}
	x = \begin{pmatrix} x_1 \\ x_2 \\ \vdots \\ x_n \end{pmatrix}, \quad 
	f(t) = \begin{pmatrix} f_1(t) \\ f_2(t) \\ \vdots \\ f_n(t) \end{pmatrix}, \quad
	A(t) = \begin{pmatrix} a_{11}(t) & a_{12}(t) & \cdots & a_{1n}(t) \\ a_{21}(t) & a_{22}(t) & \cdots & a_{2n}(t) \\ \vdots & \vdots & \ddots & \vdots \\ a_{n1}(t) & a_{n2}(t) & \cdots & a_{nn}(t) \end{pmatrix},
\end{equation*}
то лінійну неоднорідну систему можна переписати у вигляді
\begin{equation*}
	\dot x = A(t) x + f(t),
\end{equation*}
а лінійну однорідну систему у вигляді
\begin{equation*}
	\dot x = A(t) x.
\end{equation*}

Якщо функції $a_{ij}(t)$, $f_i(t)$, $i,j=\overline{1,n}$ неперервні в околі точки \[(x_0, t_0) = (x_1^0, x_2^0, \ldots, x_n^0, t_0),\] то виконані умови теореми існування та єдиності розв'язку задачі Коші, і існує єдиний розв'язок
\begin{equation*}
	x_1 = x_1(t), \quad x_2 = x_2(t), \quad \ldots, \quad x_n = x_n(t),
\end{equation*}
системи рівнянь, що задовольняє початковим даним
 \begin{equation*}
	x_1(t_0) = x_1^0, \quad x_2(t_0) = x_2^0, \quad \ldots, \quad x_n(t_0) = x_n^0.
\end{equation*}
