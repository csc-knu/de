Диференціальне рівняння першого порядку, не розв’язане відносно похідної, має такий вигляд
\begin{equation}
	\label{eq:1.5.1}
	F(x, y, y') = 0. 	
\end{equation}

\subsubsection{Частинні випадки рівнянь, що інтегруються в квадратурах}

Розглянемо ряд диференціальних рівнянь, що інтегруються в квадратурах.
\begin{enumerate}
	\item Рівняння вигляду 
	\begin{equation}
		\label{eq:1.5.2}
		F(y') = 0.
	\end{equation}
	Нехай алгебраїчне рівняння $F(k) = 0$ має принаймні один дійсний корінь $k = k_0$. Тоді, інтегруючи $y' = k_0$, одержимо $y = k_0 \cdot x + C$. Звідси $k_0 = (y - C) / x$ і вираз
	\begin{equation}
		\label{eq:1.5.3}
		F \left( \frac{y - c}{x} \right) = 0	
	\end{equation}
	містить всі розв’язки вихідного диференціального рівняння.
	\item Рівняння вигляду 
	\begin{equation}
		\label{eq:1.5.4}
		F(x, y') = 0.
	\end{equation}
	Нехай це рівняння можна записати у параметричному вигляді
	\begin{equation}
		\label{eq:1.5.5}
		\left\{\begin{aligned}
			x &= \phi(t), \\
			y' &= \psi(t).
		\end{aligned}\right.
	\end{equation}
	Використовуючи співвідношення $\diff y = y ' \cdot \diff x$, одержимо 
	\begin{equation}
		\label{eq:1.5.6}
		\diff y = \psi(t) \cdot \phi'(t) \cdot \diff t.
	\end{equation}
	Проінтегрувавши, запишемо
	\begin{equation}
		\label{eq:1.5.7}
		y = \int \psi(t) \cdot \phi'(t) \cdot \diff t + C.
	\end{equation}
	І загальний розв’язок в параметричній формі має вигляд
	\begin{equation}
		\label{eq:1.5.8}
		\left\{\begin{aligned}
		x &= \phi(t), \\
		y &= \int \psi(t) \cdot \phi'(t) \cdot \diff t + C.
		\end{aligned}\right.
	\end{equation}
	\item Рівняння вигляду 
	\begin{equation}
		\label{eq:1.5.9}
		F(y, y') = 0.
	\end{equation}
	Нехай це рівняння можна записати у параметричному вигляді
	\begin{equation}
		\label{eq:1.5.10}
		\left\{\begin{aligned}
			y &= \phi(t), \\
			y' &= \psi(t).
		\end{aligned}\right.
	\end{equation}
	Використовуючи співвідношення $\diff y = y ' \cdot \diff x$, одержимо 
	\begin{equation}
		\label{eq:1.5.11}
		\phi'(t) \cdot \diff t = \psi(t) \cdot \diff x
	\end{equation}
	і
	\begin{equation}
		\label{eq:1.5.12}
		\diff x = \frac{\phi'(t)}{\psi(t)} \cdot \diff t
	\end{equation}
	Проінтегрувавши, запишемо
	\begin{equation}
		\label{eq:1.5.13}
		x = \int \frac{\phi'(t)}{\psi(t)}\cdot \diff t + C.
	\end{equation}
	І загальний розв’язок в параметричній формі має вигляд
	\begin{equation}
		\label{eq:1.5.14}
		\left\{\begin{aligned}
		x &= \int \frac{\phi'(t)}{\psi(t)}\cdot \diff t + C, \\
		y &= \phi(t).
		\end{aligned}\right.
	\end{equation}
	\item Рівняння Лагранжа
	\begin{equation}
		\label{eq:1.5.15}
		y = \phi(y') \cdot x + \psi(y').
	\end{equation}
	Введемо параметр $y' = \frac{\diff y}{\diff x} = p$ і отримаємо
	\begin{equation}
		\label{eq:1.5.16}
		y = \phi(p) \cdot x + \psi(p).
	\end{equation}
	Продиференціювавши, запишемо
	\begin{equation}
		\label{eq:1.5.17}
		\diff y = \phi'(p) \cdot x \cdot \diff p + \phi(p) \cdot \diff x + \psi'(p) \cdot \diff p.
	\end{equation}
	Замінивши $\diff y = p \cdot \diff x$ одержимо
	\begin{equation}
		\label{eq:1.5.18}
		p \cdot \diff x = \phi'(p) \cdot x \cdot \diff p + \phi(p) \cdot \diff x + \psi'(p) \cdot \diff p.
	\end{equation}
	Звідси
	\begin{equation}
		\label{eq:1.5.19}
		(p - \phi(p)) \cdot \diff x - \phi'(p) \cdot x \cdot \diff p = \psi'(p) \cdot \diff p.
	\end{equation}
	І отримали лінійне неоднорідне диференціальне рівняння
	\begin{equation}
		\label{eq:1.5.20}
		\frac{\diff x}{\diff p} + \frac{\phi'(p)}{\phi(p)-p} \cdot x = \frac{\phi'(p)}{p-\phi(p)}.
	\end{equation}
	Його розв’язок
	\begin{multline}
		\label{eq:1.5.21}
		x = \exp\left\{\int \frac{\phi'(p)}{p-\phi(p)} \cdot \diff p\right\} \cdot \\
		\cdot \left(\int \frac{\phi'(p)}{p-\phi(p)} \cdot \exp\left\{\int \frac{\phi'(p)}{\phi(p)-p} \cdot \diff p\right\} \diff p + C \right) = \\
		= \Psi(p, C).
	\end{multline}
	І остаточний розв’язок рівняння Лагранжа в параметричній формі запишеться у вигляді
	\begin{equation}
		\label{eq:1.5.22}
		\left\{\begin{aligned}
			x &= \Psi(p,C), \\
			y &= \phi(p) \cdot \Phi(p, C) + \psi(p).
		\end{aligned}\right.
	\end{equation}
	\item Рівняння Клеро. \\

	Частинним випадком рівняння Лагранжа, що відповідає $\phi(y') = y'$ є рівняння Клеро
 	\begin{equation}
 		\label{eq:1.5.23}
 		y = y' x + \psi(y').
 	\end{equation}
	Поклавши $y' = \frac{\diff y}{\diff x} = p$, отримаємо $y = p x + \psi(p)$. Продиференціюємо 
	\begin{equation}
		\label{eq:1.5.24}
		\diff y = p \cdot \diff x + x \cdot \diff p + \psi'(p) \cdot \diff p.
	\end{equation}
	Оскільки $\diff y = p \cdot \diff x$, то
	\begin{equation}
		\label{eq:1.5.25}
		p \cdot \diff x = p \cdot \diff x + x \cdot \diff p + \psi'(p) \cdot \diff p.
	\end{equation}
	Скоротивши, одержимо
	\begin{equation}
		\label{eq:1.5.25}
		(x + \psi'(p)) \cdot \diff p = 0.
	\end{equation}
	Можливі два випадки.
	\begin{enumerate}
		\item $x + \psi'(p) - 0$ і розв’язок має вигляд
		\begin{equation}
			\label{eq:1.5.26}
			\left\{\begin{aligned}
				x &= - \psi'(p), \\
				y &= -p \cdot \psi'(p) + \psi(p).
			\end{aligned}\right.
		\end{equation}
		\item $\diff p = 0$, $p = C$ і розв’язок має вигляд
		\begin{equation}
			\label{eq:1.5.27}
			y = C x + \psi(C).
		\end{equation}
	\end{enumerate}
	Загальним розв’язком рівняння Клеро буде сім’я ``прямих'' \eqref{eq:1.5.27}. Цю сім’ю огинає особлива крива \eqref{eq:1.5.26}.
	\item Параметризація загального вигляду. Нехай диференціальне рівняння $F(x, y, y') = 0$ вдалося записати у вигляді системи рівнянь з двома параметрами
	\begin{equation}
		\label{eq:1.5.28}
		x = \phi(u, v), \quad y = \psi(u, v), \quad y' = \theta(u, v).	
	\end{equation}
	Використовуючи співвідношення $\diff y = y' \cdot \diff x$, одержимо
	\begin{multline}
		\label{eq:1.5.29}
		\frac{\partial \psi(u,v)}{\partial u} \cdot \diff u + \frac{\partial \psi(u, v)}{\partial v} \cdot \diff v = \\
		= \theta(u,v) \cdot \left( \frac{\partial \phi(u,v)}{\partial u} \cdot \diff u + \frac{\partial \phi(u, v)}{\partial v} \cdot \diff v\right)
	\end{multline}
	Перегрупувавши члени, одержимо
	\begin{multline}
		\label{eq:1.5.30}
		\left( \frac{\partial \psi(u,v)}{\partial u} - \theta(u, v) \cdot \frac{\partial \phi(u,v)}{\partial u} \right) \diff u = \\
		= \left( \theta(u,v) \cdot \frac{\partial \phi(u, v)}{\partial v} - \frac{\partial \psi(u, v)}{\partial v} \right) \diff v.
	\end{multline}
	Звідси
	\begin{equation}
		\label{eq:1.5.32}
		\frac{\diff u}{\diff v} = \frac{\theta(u,v) \cdot \frac{\partial \phi(u, v)}{\partial v} - \frac{\partial \psi(u, v)}{\partial v}}{\frac{\partial \psi(u,v)}{\partial u} - \theta(u, v) \cdot \frac{\partial \phi(u,v)}{\partial u}}.
	\end{equation}
	Або отримали рівняння вигляду
	\begin{equation}
		\label{eq:1.5.33}
		\frac{\diff u}{\diff v} = f(u, v).
	\end{equation}
	Параметризація загального вигляду не дає інтеграл диференціального рівняння. Вона дозволяє звести диференціальне рівняння, не розв’язане відносно похідної, до диференціального рівняння, розв’язаного відносно похідної.
	\item Нехай рівняння $F(x, y, y') = 0$ можна розв’язати відносно $y'$ і воно має $n$ коренів, тобто його  можна записати у вигляді  
	\begin{equation}
		\label{eq:1.5.34}
		\prod_{i=1}^n (y' - f_i(x, y)) = 0.
	\end{equation}
	Розв’язавши кожне з рівнянь $y' = f_i(x, y)$, $i=\overline{1,n}$, отримаємо $n$ загальних розв’язків (або інтервалів) $y = \phi_i(x, C)$, $i=\overline{1,n}$ (або $\phi_u(x,y)=C$, $i=\overline{1,n}$). І загальний розв’язок вихідного рівняння, не розв’язаного відносно похідної має вигляд
	\begin{equation}
		\label{eq:1.5.35}
		\prod_{i=1}^n (y - \phi_i(x, C)) = 0,
	\end{equation}
	або
	\begin{equation}
		\label{eq:1.5.36}
		\prod_{i=1}^n (\phi_i(x, y) - C) = 0.
	\end{equation}
\end{enumerate}

\subsubsection{Вправи для самостійної роботи}

\begin{enumerate}

\item Розв’язати рівняння вигляду $F(y') = 0$:
\begin{example}
	$(y')^3 - 1 = 0$;
\end{example}
\begin{solution}
	Рівняння має дійсний розв’язок, тобто воно поставлене коректно. Тому його розв’язком буде $\left(\frac{y - C}{x}\right)^3 - 1 = 0$.	
\end{solution}
\begin{multicols}{2}
\begin{problem}
	\[(y')^2 - 2 y' + 1 = 0;\]
\end{problem}
\begin{problem}
	\[ (y')^4 - 16 = 0. \]
\end{problem}
\end{multicols}

\item Розв’язати рівняння вигляду $F(x, y') = 0$:
\begin{example}
	$x = (y')^3 + y'$;
\end{example}
\begin{solution}
	Робимо параметризацію $y' = t$, $x = t^3 + t$. Використовуючи основну форму запису $\diff y = y' \cdot \diff x$ одержимо \[ \diff y = t \cdot (3t^2 + 1) \cdot \diff t.\]
	Звідси  \[ y = \int t \cdot (3t^2 + 1) \cdot \diff t = \frac{3t^4}{4} + \frac{t^2}{2} + C.\]
	Остаточний розв’язок у параметричній формі має вигляд\[ x = t^3 + t, \quad y = \frac{3t^4}{4} + \frac{t^2}{2} + C.\]
\end{solution}
\begin{multicols}{2}
\begin{problem}
	\[ x \cdot ((y')^2 - 1) = 2y'; \]
\end{problem}
\begin{problem}
	\[ x = y' \cdot \sqrt{(y')^2 - 1}; \]
\end{problem}
\begin{problem}
 	\[ y' \cdot (x - \ln y') - 1. \]
\end{problem}
\end{multicols}

\item Розв’язати рівняння вигляду $F(y, y') = 0$:
\begin{example}
	$y = (y')^2 + 2 (y')^3$;
\end{example}
\begin{solution}
	Робимо параметризацію $y' = t$, $y = t^2 + 2t^3$. Використовуючи основну форму запису $\diff y = y' \cdot \diff x$, одержуємо
	\[ (2t + 6t^2) \cdot \diff t = t \cdot \diff x.\]
	Звідси \[ \diff x = (2 + 6t) \cdot \diff t, \quad x = \int(2+6t) \cdot \diff t = 2t + 3t^2 + C.\]
	Остаточний розв’язок у параметричній формі має вигляд 
	\[ x = 2t + 3t^2, \quad y = t^2 + 2t^.\]
	Крім того за рахунок скорочення втрачено $y \equiv 0$.
\end{solution}
\begin{multicols}{2}
\begin{problem}
	\[ y = \ln ( 1 + (y')^2); \]
\end{problem}
\begin{problem}
	\[ y = (y' - 1) \cdot e^{y'}; \]
\end{problem}
\begin{problem}
 	\[ (y')^4 - (y')^2 = y^2. \]
\end{problem}
\end{multicols}

\item Розв’язати рівняння Лагранжа
\begin{example}
	$y = - x y' + 4 \sqrt{y'}$;
\end{example}
\begin{solution}
	Робимо параметризацію $y' = t$, $y = - xt + 4 \sqrt{t}$. Диференціюємо друге рівняння.
	\[ \diff y = - x \cdot \diff t - t \cdot \diff x + 2 \cdot \frac{1}{\sqrt{t}} \cdot \diff t.\]
	Оскільки зроблено заміну $\diff y = t \cdot \diff x$, то одержимо
	\[ t \cdot \diff x = - x \cdot \diff t - t \cdot \diff x + 2 \cdot \frac{1}{\sqrt{t}} \cdot \diff t,\]
	або
	\[ 2 t \cdot \diff x = - x \cdot \diff t + 2 \cdot \frac{1}{\sqrt{t}} \cdot \diff t.\]
	Звідси
	\[ \frac{\diff x}{\diff t} + \frac{x}{2t} = \frac{1}{t \cdot \sqrt{t}}.\]
	Розв’язок лінійного неоднорідного рівняння може бути представлений у вигляді
	\begin{multline*}
		x = \exp\left\{ - \int \frac{\diff t}{2t} \right\} \cdot \left( \int \exp\left\{\int \frac{\diff t}{2t} \right\} \cdot \frac{1}{t\cdot\sqrt{t}} + C\right) = \\
		= \frac{1}{\sqrt{t}} \cdot \left(\int \frac {\diff t}{t} + C\right) = \frac{\ln |t| + C}{\sqrt{t}}.
	\end{multline*}
	Остаточно маємо \[ x = \frac{\ln |t| + C}{\sqrt{t}}, \quad y = - \sqrt{t} \cdot (\ln |t| + C) + 4 \sqrt{t}. \] Крім того при діленні на $t$ втратили $y \equiv 0$.
\end{solution}
\begin{multicols}{2}
\begin{problem}
	\[ y = 2 x y' - 4 \cdot (y')^3;\]
\end{problem}
\begin{problem}
	\[ y = x \cdot (y')^2 - 2 \cdot (y')^3;\]
\end{problem}
\begin{problem}
 	\[ x y' \cdot (y' + 2) = y;\]
\end{problem}
\begin{problem}
 	\[ 2 x y' - y = \ln y'.\]
\end{problem}
\end{multicols}

\item Розв’язати рівняння Клеро
\begin{example}
	$y = x y' - (y')^2$;
\end{example}
\begin{solution}
	Робимо параметризацію $y' = t$, $y = x t - t^2$. Диференціюємо друге рівняння:
	\[ \diff y = x \cdot \diff t + t \cdot \diff x - 2 t \cdot \diff t.\]
	Оскільки зроблено заміну $\diff y = t \cdot \diff x$, то одержимо
	\[ t \cdot \diff x = x \cdot \diff t + t \cdot \diff x - 2 t \cdot \diff t.\]
	Звідси $(x - 2t) \cdot \diff t = 0$. І маємо дві гілки
	\begin{enumerate}
		\item Особливий розв’язок $x = 2t$, $y = t^2$, або $y = \frac{x^2}{4}$.
		\item Загальний розвозок $y = C x - C^2$.
	\end{enumerate}
\end{solution}
\begin{multicols}{2}
\begin{problem}
	\[ y=xy'+4\sqrt{y'}; \]
\end{problem}
\begin{problem}
	\[ y=xy'+2-y'; \]
\end{problem}
\begin{problem}
	\[ y=xy'-\ln y'; \]
\end{problem}
\begin{problem}
	\[ y=xy'+\sin y'; \]
\end{problem}
\begin{problem}
	\[ y=xy'+\sqrt{1+(y')^2}; \]
\end{problem}
\begin{problem}
	\[ y=xy'+(y')^3; \]
\end{problem}
\begin{problem}
	\[ y=xy'+\cos(2+y'); \]
\end{problem}
\begin{problem}
	\[ y=xy'-\ln\sqrt{1+(y')^2}; \]
\end{problem}
\begin{problem}
	\[ y=xy'-y'-(y')^3; \]
\end{problem}
\begin{problem}
	\[ y=xy'-\sqrt{2-(y')^2}; \]
\end{problem}
\begin{problem}
	\[ y=xy'+\sqrt{2y'+2}; \]
\end{problem}
\begin{problem}
	\[ y=xy'-e^{y'}; \]
\end{problem}
\begin{problem}
	\[ y=xy'-\tan y'; \]
\end{problem}
\begin{problem}
	\[ (y')^3=3(xy'-y). \]
\end{problem}
\end{multicols}

\item Розв’язати рівняння параметризацією загального виду
\begin{example}
	$(y')^2 - 2 x y' = x^2 - 4y$;
\end{example}
\begin{solution}
	Введемо параметризацію рівняння \[x = u, \quad y' = v, \quad y = \frac {u^2 + 2 u v - v^2}{4}.\] 

	Використовуючи співвідношення $\diff y = y' \cdot \diff x$, одержимо рівняння
	\[ \frac1u (2u \cdot \diff u + 2 u \cdot \diff v + 2 v \cdot \diff u - 2 v \cdot \diff v) = v \cdot \diff u.\]

	Перепишемо його у вигляді \[ (u+v)\cdot\diff u+(u-v)\cdot\diff v=2v\cdot\diff u,\] або \[ (u-v)\cdot\diff u+(u-v)\cdot\diff v=0,\]
	Воно розділяється на два 
	\begin{enumerate}
		\item $\diff u + \diff v = 0 \implies v = - u + C$. \\

		Підставивши в параметризовану систему, одержуємо \[x = u, \quad y = \frac{u^2+2u\cdot(-u+C)-(-u+C)^2}{4},\] або \[y = \frac{x^2+2x\cdot(-x+C)-(-x+C)^2}{4} = \frac{-2x^2+4Cx-C^2}{4}.\]
		\item $u - v = 0 \implies v = u$. І розв’язок має вигляд $y = \frac{x^2}{2}$.
	\end{enumerate}
\end{solution}

\begin{multicols}{2}
\begin{problem}
	\[5y+(y')^2=x\cdot(x+y');\]
\end{problem}
\begin{problem}
	\[x^2\cdot(y')^2=xyy'+1;\]
\end{problem}
\begin{problem}
	\[(y')^3+y^2=xyy';\]
\end{problem}
\begin{problem}
	\[y=x\cdot(y')^2-2\cdot(y')^3;\]
\end{problem}
\begin{problem}
	\[2xy'-y=y'\cdot\ln(yy');\]
\end{problem}
\begin{problem}
	\[y'=e^{xy'/y}.\]
\end{problem}
\end{multicols}

\item Розв’язати рівняння
\begin{example}
	$(y')^2 - y^2 = 0$;
\end{example}
\begin{solution}
	Це рівняння розв’язується відносно $y'$. Маємо  \[y' = y, \quad y' = - y.\] Розв’язок першого має вигляд $y = ce^x$, другого $Ce^{-x}$. Загальний розв’язок має вигляд \[ (y - ce^x)\cdot(y-Ce^{-x})=0.\]
\end{solution}
\begin{multicols}{2}
\begin{problem}
	\[y^2\cdot((y')^2+1)=1;\]
\end{problem}
\begin{problem}
	\[(y')^2-4y^3=0;\]
\end{problem}
\begin{problem}
	\[x\cdot(y')^2=y;\]
\end{problem}
\begin{problem}
	\[(y')^2+xy=y^2+xy';\]
\end{problem}
\begin{problem}
	\[xy'\cdot(xy'+y)=2y^2.\]
\end{problem}
\end{multicols}
\end{enumerate}