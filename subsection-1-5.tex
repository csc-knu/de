Диференціальне рівняння першого порядку, не розв’язане відносно похідної, має такий вигляд
\begin{equation}
	\label{eq:1.5.1}
	F(x, y, y') = 0. 	
\end{equation}

\subsubsection{Частинні випадки рівнянь, що інтегруються в квадратурах}

Розглянемо ряд диференціальних рівнянь, що інтегруються в квадратурах.
\begin{enumerate}
	\item Рівняння вигляду 
	\begin{equation}
		\label{eq:1.5.2}
		F(y') = 0.
	\end{equation}
	Нехай алгебраїчне рівняння $F(k) = 0$ має принаймні один дійсний корінь $k = k_0$. Тоді, інтегруючи $y' = k_0$, одержимо $y = k_0 \cdot x + C$. Звідси $k_0 = (y - C) / x$ і вираз
	\begin{equation}
		\label{eq:1.5.3}
		F \left( \frac{y - c}{x} \right) = 0	
	\end{equation}
	містить всі розв’язки вихідного диференціального рівняння.
	\item Рівняння вигляду 
	\begin{equation}
		\label{eq:1.5.4}
		F(x, y') = 0.
	\end{equation}
	Нехай це рівняння можна записати у параметричному вигляді
	\begin{equation}
		\label{eq:1.5.5}
		\left\{\begin{aligned}
			x &= \phi(t), \\
			y' &= \psi(t).
		\end{aligned}\right.
	\end{equation}
	Використовуючи співвідношення $\diff y = y ' \cdot \diff x$, одержимо 
	\begin{equation}
		\label{eq:1.5.6}
		\diff y = \psi(t) \cdot \phi'(t) \cdot \diff t.
	\end{equation}
	Проінтегрувавши, запишемо
	\begin{equation}
		\label{eq:1.5.7}
		y = \int \psi(t) \cdot \phi'(t) \cdot \diff t + C.
	\end{equation}
	І загальний розв’язок в параметричній формі має вигляд
	\begin{equation}
		\label{eq:1.5.8}
		\left\{\begin{aligned}
		x &= \phi(t), \\
		y &= \int \psi(t) \cdot \phi'(t) \cdot \diff t + C.
		\end{aligned}\right.
	\end{equation}
	\item Рівняння вигляду 
	\begin{equation}
		\label{eq:1.5.9}
		F(y, y') = 0.
	\end{equation}
	Нехай це рівняння можна записати у параметричному вигляді
	\begin{equation}
		\label{eq:1.5.10}
		\left\{\begin{aligned}
			y &= \phi(t), \\
			y' &= \psi(t).
		\end{aligned}\right.
	\end{equation}
	Використовуючи співвідношення $\diff y = y ' \cdot \diff x$, одержимо 
	\begin{equation}
		\label{eq:1.5.11}
		\phi'(t) \cdot \diff t = \psi(t) \cdot \diff x
	\end{equation}
	і
	\begin{equation}
		\label{eq:1.5.12}
		\diff x = \frac{\phi'(t)}{\psi(t)} \cdot \diff t
	\end{equation}
	Проінтегрувавши, запишемо
	\begin{equation}
		\label{eq:1.5.13}
		x = \int \frac{\phi'(t)}{\psi(t)}\cdot \diff t + C.
	\end{equation}
	І загальний розв’язок в параметричній формі має вигляд
	\begin{equation}
		\label{eq:1.5.14}
		\left\{\begin{aligned}
		x &= \int \frac{\phi'(t)}{\psi(t)}\cdot \diff t + C, \\
		y &= \phi(t).
		\end{aligned}\right.
	\end{equation}
	\item Рівняння Лагранжа
	\begin{equation}
		\label{eq:1.5.15}
		y = \phi(y') \cdot x + \psi(y').
	\end{equation}
	Введемо параметр $y' = \frac{\diff y}{\diff x} = p$ і отримаємо
	\begin{equation}
		\label{eq:1.5.16}
		y = \phi(p) \cdot x + \psi(p).
	\end{equation}
	Продиференціювавши, запишемо
	\begin{equation}
		\label{eq:1.5.17}
		\diff y = \phi'(p) \cdot x \cdot \diff p + \phi(p) \cdot \diff x + \psi'(p) \cdot \diff p.
	\end{equation}
	Замінивши $\diff y = p \cdot \diff x$ одержимо
	\begin{equation}
		\label{eq:1.5.18}
		p \cdot \diff x = \phi'(p) \cdot x \cdot \diff p + \phi(p) \cdot \diff x + \psi'(p) \cdot \diff p.
	\end{equation}
	Звідси
	\begin{equation}
		\label{eq:1.5.19}
		(p - \phi(p)) \cdot \diff x - \phi'(p) \cdot x \cdot \diff p = \psi'(p) \cdot \diff p.
	\end{equation}
	І отримали лінійне неоднорідне диференціальне рівняння
	\begin{equation}
		\label{eq:1.5.20}
		\frac{\diff x}{\diff p} + \frac{\phi'(p)}{\phi(p)-p} \cdot x = \frac{\phi'(p)}{p-\phi(p)}.
	\end{equation}
	Його розв’язок
	\begin{multline}
		\label{eq:1.5.21}
		x = \exp\left\{\int \frac{\phi'(p)}{p-\phi(p)} \cdot \diff p\right\} \cdot \\
		\cdot \left(\int \frac{\phi'(p)}{p-\phi(p)} \cdot \exp\left\{\int \frac{\phi'(p)}{\phi(p)-p} \cdot \diff p\right\} \diff p + C \right) = \\
		= \Psi(p, C).
	\end{multline}
	І остаточний розв’язок рівняння Лагранжа в параметричній формі запишеться у вигляді
	\begin{equation}
		\label{eq:1.5.22}
		\left\{\begin{aligned}
			x &= \Psi(p,C), \\
			y &= \phi(p) \cdot \Phi(p, C) + \psi(p).
		\end{aligned}\right.
	\end{equation}
\end{enumerate}
