Нехай маємо дробово-лінійне рівняння вигляду
\begin{equation*}
	\frac{\diff y}{\diff x} = f \left( \frac{a_1 x + b_1 y + c_1}{a_2 x + b_2 y + c_2} \right).
\end{equation*}

Розглянемо два випадки
\begin{enumerate}
	\item 
	\begin{equation*}
		\Delta = 
		\begin{vmatrix} 
			a_1 & b_1 \\ 
			a_2 & b_2 
		\end{vmatrix} 
		\ne 0.
	\end{equation*}

	Тоді система алгебраїчних рівнянь
	\begin{equation*}
		\left\{
			\begin{aligned}
				a_1 x + b_1 y + c_1 &= 0, \\
				a_2 x + b_2 y + c_2 &= 0,
			\end{aligned}
		\right.
	\end{equation*}
	має єдиний розв’язок $(x_0, y_0)$. Проведемо заміну 
	\begin{equation*}
		\left\{
			\begin{aligned}
			x &= x_1 + x_0, \\
			y &= y_1 + y_0
			\end{aligned}
		\right.
	\end{equation*}
	та отримаємо
	\begin{multline*}
		\frac{\diff y_1}{\diff x_1} = f \left( \frac{a_1 (x_1 + x_0) + b_1 (y_1 + y_0) + c_1}{a_2 (x_1 + x_0) + b_2 (y_1 + y_0) + c_2} \right) = \\
		= f \left( \frac{a_1 x_1 + b_1 y_1 + (a_1 x_0 + b_1 y_0 + c_1)}{a_2 x_1 + b_2 y_1 + (a_2 x_0 + b_2 y_0 + c_2)} \right)
	\end{multline*}

	Оскільки $(x_0, y_0)$ --- розв’язок алгебраїчної системи, то диференціальне рівняння набуде вигляду
	\begin{equation*}
		\frac{\diff y_1}{\diff x_1} = f \left( \frac{a_1 x_1 + b_1 y_1}{a_2 x_1 + b_2 y_1} \right)
	\end{equation*}
	і є однорідним нульового ступеня. Робимо заміну 
	\begin{equation*}
		y_1 = u x_1, \quad \diff y_1 = u \diff x_1 + x_1 \diff u.
	\end{equation*}

	Підставимо в рівняння
	\begin{equation*}
		u + x_1 \frac{\diff u}{\diff x_1} = f \left( \frac{a_1 x_1 + b_1 u x_1}{a_2 x_1 + b_2 u x_1} \right).
	\end{equation*}
	
	Одержимо
	\begin{equation*}
		x_1 \diff u + \left( u - f \left( \frac{a_1 x_1 + b_1 u x_1}{a_2 x_1 + b_2 u x_1} \right) \right) \diff x_1 = 0.
	\end{equation*}

	Розділивши змінні, маємо
	\begin{equation*}
		\int \frac{\diff u}{u - f \left( \frac{a_1 x_1 + b_1 u x_1}{a_2 x_1 + b_2 u x_1} \right)} + \ln (x_1) = C.
	\end{equation*}

	І загальний інтеграл рівняння має вигляд $\Phi(u, x_1) = C$. Повернувшись до вихідних змінних, запишемо
	\begin{equation*}
		\Phi \left( \frac{y - y_0}{x - x_0}, x - x_0 \right) = C.
	\end{equation*}

	\item Нехай 
	\begin{equation*}
		\Delta = 
		\begin{vmatrix} 
			a_1 & b_1 \\ 
			a_2 & b_2 
		\end{vmatrix} 
		= 0,
	\end{equation*}
	тобто коефіцієнти рядків лінійно залежні і
	\begin{equation*}
		a_1 x + b_1 y = \alpha (a_2 x + b_2 y).
	\end{equation*}

	Робимо заміну $a_2 x + b_2 y = z$. Звідси 
	\begin{equation*}
		\frac{\diff y}{\diff x} = \frac{1}{b_2} \left( \frac{\diff z}{\diff x} - a_2 \right)
	\end{equation*}

	Підставивши в диференціальне рівняння, одержимо
	\begin{equation*}
		\frac{1}{b_2} \left( \frac{\diff z}{\diff x} - a_2 \right) = f \left ( \frac{\alpha z + c_1}{z + c_2} \right),
	\end{equation*}
	або
	\begin{equation*}
		\frac{\diff z}{\diff x} = a_2 + b_2 f \left ( \frac{\alpha z + c_1}{z + c_2} \right),
	\end{equation*}

	Розділивши змінні, отримаємо
	\begin{equation*}
		\int \frac{\diff z}{a_2 + b_2 f \left ( \frac{\alpha z + c_1}{z + c_2} \right)} - x = C,
	\end{equation*}

	Загальний інтеграл має вигляд $\Phi(a_2 x + b_2 y, x) = C$.
\end{enumerate}
