Рівняння першого порядку, що розв’язане відносно похідної, має вигляд
\begin{equation}
	\label{eq:1.1.1}
	\frac{\diff y}{\diff x} = f(x, y).	
\end{equation}

Диференціальне рівняння становить зв’язок між координатами точки та кутовим коефіцієнтом дотичної $\frac{\diff y}{\diff x}$ до графіку розв’язку в цій же точці. Якщо знати $x$ та $y$, то можна обчислити $f(x, y)$ тобто $\frac{\diff y}{\diff x}$. Таким чином, диференціальне рівняння визначає поле напрямків, і задача інтегрування рівнянь зводиться до знаходження кривих, що звуться інтегральними кривими, напрям дотичних до яких в кожній точці співпадає з напрямом поля.

\subsection{Рівняння зі змінними, що розділяються}

\subsubsection{Загальна теорія}

Рівняння вигляду
\begin{equation}
	\label{eq:1.1.2}
	\frac{\diff y}{\diff x} = f(x) \cdot g(y),
\end{equation}
або більш загального вигляду
\begin{equation}
	\label{eq:1.1.3}
	f_1(x) \cdot f_2(y) \cdot \diff x + g_1(x) \cdot g_2(y) \cdot \diff y = 0
\end{equation}
називаються рівняннями зі змінними, що розділяються. Розділимо його на $f_2(y) \cdot g_1(x)$ і одержимо
\begin{equation}
	\label{eq:1.1.4}
	\frac{f_1(x)}{g_1(x)} \cdot \diff x + \frac{g_2(y)}{f_2(y)} \cdot \diff y = 0.
\end{equation}
Взявши інтеграли, отримаємо
\begin{equation}
	\label{eq:1.1.5}
	\int \frac{f_1(x)}{g_1(x)} \cdot \diff x + \int \frac{g_2(y)}{f_2(y)} \cdot \diff y = C,
\end{equation}
або 
\begin{equation}
	\label{eq:1.1.6}
	\Phi(x, y) = C.
\end{equation}

\begin{definition}
	Кінцеве рівняння \eqref{eq:1.1.6}, що визначає розв’язок диференціального рівняння як неявну функцію від $x$, називається інтегралом розглянутого рівняння.
\end{definition}

\begin{definition}
	Рівняння \eqref{eq:1.1.6}, що визначає всі без винятку розв’язки даного диференціального рівняння, називається загальним інтегралом.
\end{definition}

Бувають випадки (в основному), що невизначені інтеграли з \eqref{eq:1.1.4} не можна записати в елементарних функціях. Незважаючи на це, задача інтегрування вважається виконаною. Кажуть, що диференціальне рівняння розв’язане в квадратурах. \\

Можливо, що загальний інтеграл розв’язується відносно $y$: 
\begin{equation}
	\label{eq:1.1.7}
	y = y(x, C).
\end{equation} Тоді, завдяки вибору $C$, можна одержати всі розв’язки. 

\begin{definition}
	Залежність \eqref{eq:1.1.7}, що тотожньо задовольняє вихідному диференціальному рівнянню, де $C$ -- довільна стала, називається загальним розв’язком диференціального рівняння.
\end{definition}

Геометрично загальний розв’язок являє собою сім’ю кривих, що не перетинаються, які заповнюють деяку область. Іноді треба виділити одну криву сім’ї, що проходить через задану точку $M(x_0, y_0)$.

\begin{definition}
	Знаходження розв’язку $y = y(x)$, що проходить через задану точку $M(x_0, y_0)$, називається розв’язком задачі Коші.
\end{definition}

\begin{definition}
	Розв’язок, який записаний у вигляді $y = y(x, x_0, y_0)$ і задовольняє умові $y(x, x_0, y_0) = y_0$, називається розв’язком у формі Коші.
\end{definition}