Рівняння вигляду
\begin{equation*}
	%\label{eq:3.1}
	a_0(x) \cdot y^{(n)} + a_1(x) \cdot y^{(n - 1)} + \ldots + a_n(x) \cdot y = b(x)
\end{equation*} 
називається лінійним неоднорідним диференціальним рівнянням $n$-го порядку. \\

Рівняння вигляду
\begin{equation*}
	%\label{eq:3.2}
	a_0(x) \cdot y^{(n)} + a_1(x) \cdot y^{(n - 1)} + \ldots + a_n(x) \cdot y = 0
\end{equation*} 
називається лінійним однорідним диференціальним рівнянням $n$-го порядку. \\

Якщо при $x \in [a, b]$, $a_0(x) \ne 0$ коефіцієнти $b(x)$, $a_i(x)$, $i=\overline{0,n}$ неперервні, то для рівняння
\begin{equation*}
	%\label{eq:3.3}
	y^{(n)} = - \frac{a_1(x)}{a_0(x)} \cdot y^{(n - 1)} - \ldots - \frac{a_n(x)}{a_0(x)} \cdot y + \frac{b(x)}{a_0(x)}.
\end{equation*}
виконуються умови теореми існування та єдиності і існує єдиний роз\-в'яз\-ок $y = y(x)$, що задовольняє початковим умовам
\begin{equation*}
	%\label{eq:3.4}
	y(x_0) = y_0, \quad y'(x_0) = y_0', \quad \ldots, \quad y^{(n - 1)} = y_0^{(n - 1)}.
\end{equation*}
