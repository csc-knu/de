Нехай маємо систему диференціальних рівнянь
\begin{equation*}
	\left\{
		\begin{aligned}
			\dot x_1 &= f_1 (x_1, x_2, \ldots, x_n, t), \\
			\dot x_2 &= f_2 (x_1, x_2, \ldots, x_n, t), \\
			\ldots & \ldots \ldots \ldots \ldots \ldots \ldots \ldots, \\
			\dot x_n &= f_n (x_1, x_2, \ldots, x_n, t).
		\end{aligned}
	\right.
\end{equation*}
і заданий її розв'язок $x_1 = x_1(t), x_2 = x_2(t), \ldots, x_n = x_n(t)$. Якщо цей розв'язок підставити в перше рівняння, то вийде тотожність і її можна диференціювати
\begin{equation*}
	\frac{\diff^2 x_1}{\diff t^2} = \frac{\partial f_1}{\partial t} + \sum_{i = 1}^n \frac{\partial f_1}{\partial x_i} \cdot \frac{\diff x_i(t)}{\diff t}.
\end{equation*}

Підставивши замість $\frac{\diff x_i(t)}{\diff t}$ їх значення, одержимо
\begin{equation*}
	\frac{\diff^2 x_1}{\diff t^2} = \frac{\partial f_1}{\partial t} + \sum_{i = 1}^n \frac{\partial f_1}{\partial x_i} \cdot f_i = F_2(t, x_1, x_2, \ldots, x_n).
\end{equation*}

Знову диференціюємо це рівняння й одержимо
\begin{equation*}
	\frac{\diff^3 x_1}{\diff t^3} = \frac{\partial F_2}{\partial t} + \sum_{i = 1}^n \frac{\partial F_2}{\partial x_i} \cdot \frac{\diff x_i(t)}{\diff t} = \frac{\partial F_2}{\partial t} + \sum_{i = 1}^n \frac{\partial F_2}{\partial x_i} \cdot f_i = F_3(t, x_1, x_2, \ldots, x_n).
\end{equation*}

Продовжуючи процес далі, одержимо
\begin{align*}
	\frac{\diff^{n - 1} x_1}{\diff t^{n - 1}} &= F_{n - 1}(t, x_1, x_2, \ldots, x_n), \\
	\frac{\diff^n x_1}{\diff t^n} &= F_n(t, x_1, x_2, \ldots, x_n).
\end{align*} 
 
Таким чином, маємо систему
\begin{equation*}
	\left\{
		\begin{aligned}
			\frac{\diff x_1}{\diff t} &= f_1 (x_1, x_2, \ldots, x_n, t), \\
			\frac{\diff^2 x_1}{\diff t^2} &= F_2(t, x_1, x_2, \ldots, x_n), \\
			\ldots \ldots & \ldots \ldots \ldots \ldots \ldots \ldots \ldots \ldots, \\
			\frac{\diff^{n - 1} x_1}{\diff t^{n - 1}} &= F_{n - 1}(t, x_1, x_2, \ldots, x_n), \\
			\frac{\diff^n x_1}{\diff t^n} &= F_n(t, x_1, x_2, \ldots, x_n).
		\end{aligned}
	\right.
\end{equation*}

Припустимо, що \[\frac{D(f_1, F_2, \ldots, F_{n - 1})}{D(x_2, x_3, \ldots, x_n)} \ne 0.\] Тоді систему перших $(n - 1)$ рівнянь 
\begin{equation*}
	\left\{
		\begin{aligned}
			\frac{\diff x_1}{\diff t} &= f_1 (x_1, x_2, \ldots, x_n, t), \\
			\frac{\diff^2 x_1}{\diff t^2} &= F_2(t, x_1, x_2, \ldots, x_n), \\
			\ldots \ldots & \ldots \ldots \ldots \ldots \ldots \ldots \ldots \ldots, \\
			\frac{\diff^{n - 1} x_1}{\diff t^{n - 1}} &= F_{n - 1}(t, x_1, x_2, \ldots, x_n).
		\end{aligned}
	\right.
\end{equation*}
можна розв'язати відносно останніх $(n - 1)$ змінних $x_2, x_3, \ldots, x_n$ і одержати
\begin{equation*}
	\left\{
		\begin{aligned}
			x_2 &= \phi_2 \left( t, x_1, \frac{\diff x_1}{\diff t}, \ldots, \frac{\diff^{n - 1} x_1}{\diff t^{n - 1}} \right), \\
			x_3 &= \phi_3 \left( t, x_1, \frac{\diff x_1}{\diff t}, \ldots, \frac{\diff^{n - 1} x_1}{\diff t^{n - 1}} \right), \\
			\ldots & \ldots \ldots \ldots \ldots \ldots \ldots \ldots \ldots \ldots \ldots, \\
			x_n &= \phi_n \left( t, x_1, \frac{\diff x_1}{\diff t}, \ldots, \frac{\diff^{n - 1} x_1}{\diff t^{n - 1}} \right), \\
		\end{aligned}
	\right.
\end{equation*}

Підставивши одержані вирази в останнє рівняння, запишемо
\begin{multline*}
	\frac{\diff^n x_1}{\diff t^n} = F_n \left( t, x_1, \phi_2 \left( t, x_1, \frac{\diff x_1}{\diff t}, \ldots, \frac{\diff^{n - 1} x_1}{\diff t^{n - 1}} \right), \ldots, \right. \\ \left. \phi_n \left( t, x_1, \frac{\diff x_1}{\diff t}, \ldots, \frac{\diff^{n - 1} x_1}{\diff t^{n - 1}} \right) \right).	
\end{multline*}

Або, після перетворень
\begin{equation*}
	\frac{\diff^n x_1}{\diff t^n} = \Phi \left( t, x_1, \frac{\diff x_1}{\diff t}, \ldots, \frac{\diff^{n - 1} x_1}{\diff t^{n - 1}} \right),
\end{equation*}
одержимо одне диференціальне рівняння $n$-го порядку. \\

У загальному випадку, одержимо, що система диференціальних рівнянь першого порядку
\begin{equation*}
	\left\{
		\begin{aligned}
			\dot x_1 &= f_1 (x_1, x_2, \ldots, x_n, t), \\
			\dot x_2 &= f_2 (x_1, x_2, \ldots, x_n, t), \\
			\ldots & \ldots \ldots \ldots \ldots \ldots \ldots \ldots, \\
			\dot x_n &= f_n (x_1, x_2, \ldots, x_n, t).
		\end{aligned}
	\right.
\end{equation*}
зводиться до одного рівняння $n$-го порядку
\begin{equation*}
	\frac{\diff^n x_1}{\diff t^n} = \Phi \left( t, x_1, \frac{\diff x_1}{\diff t}, \ldots, \frac{\diff^{n - 1} x_1}{\diff t^{n - 1}} \right),
\end{equation*}
і системи $(n - 1)$ рівнянь зв'язку
\begin{equation*}
	\left\{
		\begin{aligned}
			x_2 &= \phi_2 \left( t, x_1, \frac{\diff x_1}{\diff t}, \ldots, \frac{\diff^{n - 1} x_1}{\diff t^{n - 1}} \right), \\
			x_3 &= \phi_3 \left( t, x_1, \frac{\diff x_1}{\diff t}, \ldots, \frac{\diff^{n - 1} x_1}{\diff t^{n - 1}} \right), \\
			\ldots & \ldots \ldots \ldots \ldots \ldots \ldots \ldots \ldots \ldots \ldots, \\
			x_n &= \phi_n \left( t, x_1, \frac{\diff x_1}{\diff t}, \ldots, \frac{\diff^{n - 1} x_1}{\diff t^{n - 1}} \right), \\
		\end{aligned}
	\right.
\end{equation*}
 
\begin{remark}
	Було зроблене припущення, що \[\frac{D(f_1, F_2, \ldots, F_{n - 1})}{D(x_2, x_3, \ldots, x_n)} \ne 0.\] Якщо ця умова не виконана, то можна зводити до рівняння щодо інших змінних, наприклад відносно $x_2$.
\end{remark}
