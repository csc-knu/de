% version 1.0
Нехай рівняння має вигляд
\begin{equation*}
	M(x, y) \cdot \diff	x + N(x, y) \cdot \diff y = 0.
\end{equation*}

Якщо функції $M(x, y)$ та $N(x, y)$ однорідні одного ступеня, то рівняння називається однорідним. Нехай функції $M(x, y)$ та $N(x, y)$ однорідні ступеня $k$, тобто
\begin{equation*}
	M(t \cdot x, t \cdot y) = t^k \cdot M(x, y), \qquad N(t \cdot x, t \cdot y) = t^k \cdot N(x, y).
\end{equation*}

Робимо заміну 
\begin{equation*}
	y = u x, \quad \diff y = u \diff x + x \diff u.
\end{equation*}

Після підстановки одержуємо
\begin{equation*}
	M(x, u x) \cdot \diff x + N(x, u x) \cdot (u \diff x + x \diff u) = 0,
\end{equation*}
або 
\begin{equation*}
	x^k M(1, u) \cdot \diff x + x^k N(1, u) \cdot (u \diff x + x \diff u) = 0.
\end{equation*}

Скоротивши на $x^k$ і розкривши дужки, запишемо 
\begin{equation*}
	M(1, u) \cdot \diff x + N(1, u) \cdot u \diff x + N(1, u) \cdot x \diff u = 0.
\end{equation*}

Згрупувавши, одержимо рівняння зі змінними, що розділяються
\begin{equation*}
	(M(1, u) + N(1, u) \cdot u) \diff x + N(1, u) \cdot x \diff u = 0,
\end{equation*}
або 
\begin{equation*}
	\int \frac{\diff x}{x} + \int \frac{N(1, u) \cdot \diff u}{M(1, u) + N(1, u) \cdot u} = C.
\end{equation*}

Взявши інтеграли та замінивши $u = y / x$, отримаємо загальний інтеграл $\Phi(x, y / x) = C$.