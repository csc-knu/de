В деяких випадках рівняння
\begin{equation*}
	M(x, y) \diff x + N(x, y) \diff y = 0,
\end{equation*}
не є рівнянням в повних диференціалах, але існує функція $\mu = \mu(x,y)$ така, що рівняння
\begin{equation*}
	\mu(x,y) M(x, y) \diff x + \mu(x,y) N(x, y) \diff y = 0,
\end{equation*}
вже буде рівнянням в повних диференціалах. Необхідною та достатньою умовою цього є рівність
\begin{equation*}
	\frac{\partial}{\partial y} (\mu(x,y) M(x, y)) = \frac{\partial}{\partial x} (\mu(x,y) N(x, y)),
\end{equation*}
або
\begin{equation*}
	\frac{\partial \mu}{\partial y} M + \mu \frac{\partial M}{\partial y} = \frac{\partial \mu}{\partial x} N + \mu \frac{\partial N}{\partial x}.
\end{equation*}

Таким чином замість звичайного диференціального рівняння відносно функції $y(x)$ одержимо диференціальне рівняння в частинних похідних відносно функції $\mu(x, y)$. \parvskip

Задача інтегрування його значно спрощується, якщо відомо в якому вигляді шукати функцію $\mu(x,y)$, наприклад $\mu = \mu(\omega(x,y))$ де $\omega(x,y)$ --- відома функція. В цьому випадку одержуємо
\begin{equation*}
	\frac{\partial \mu}{\partial y} = \frac{\diff \mu}{\diff \omega} \cdot \frac{\partial \omega}{\partial y}, \quad \frac{\partial \mu}{\partial x} = \frac{\diff \mu}{\diff \omega} \cdot \frac{\partial \omega}{\partial x}
\end{equation*}

Після підстановки в попереднє рівняння маємо
\begin{equation*}
	\frac{\diff \mu}{\diff \omega} \cdot \frac{\partial \omega}{\partial y} \cdot M + \mu \frac{\partial M}{\partial y} = \frac{\diff \mu}{\diff \omega} \cdot \frac{\partial \omega}{\partial x} \cdot N + \mu \frac{\partial N}{\partial x}.
\end{equation*}
або
\begin{equation*}
	\frac{\diff \mu}{\diff \omega} \left( \frac{\partial \omega}{\partial x} N - \frac{\partial \omega}{\partial y} M \right) = \mu \left( \frac{\partial M}{\partial y} - \frac{\partial N}{\partial x} \right).
\end{equation*}

Розділимо змінні
\begin{equation*}
	\frac{\diff \mu}{\mu} = \frac{\frac{\partial M}{\partial y} - \frac{\partial N}{\partial x} }{\frac{\partial \omega}{\partial x} N - \frac{\partial \omega}{\partial y} M} \diff \omega.
\end{equation*}

Проінтегрувавши і поклавши сталу інтегрування одиницею, одержимо:
\begin{equation*}
	\mu(\omega(x,y)) = \exp\left\{\int \frac{\frac{\partial M}{\partial y} - \frac{\partial N}{\partial x} }{\frac{\partial \omega}{\partial x} N - \frac{\partial \omega}{\partial y} M} \diff \omega\right\}.
\end{equation*}

Розглянемо частинні випадки.
\begin{enumerate}
	\item Нехай $\omega(x, y) = x$. Тоді $\partial \omega / \partial x = 1$, $\partial \omega / \partial y = 0$, $\diff \omega = \diff x$ і формула має вигляд
	\begin{equation*}
		\mu(\omega(x,y)) = \exp\left\{\int \frac{\frac{\partial M}{\partial y} - \frac{\partial N}{\partial x} }{N} \diff x\right\}.
	\end{equation*}	

	\item Нехай $\omega(x, y) = y$. Тоді $\partial \omega / \partial x = 0$, $\partial \omega / \partial y = 1$, $\diff \omega = \diff y$ і формула має вигляд
	\begin{equation*}
		\mu(\omega(x,y)) = \exp\left\{\int \frac{\frac{\partial M}{\partial y} - \frac{\partial N}{\partial x} }{-M} \diff y\right\}.
	\end{equation*}

	\item Нехай $\omega(x, y) = x^2 \pm y^2$. Тоді $\partial \omega / \partial x = 2 x$, $\partial \omega / \partial y = \pm 2y$, $\diff \omega = \diff (x^2 \pm y^2)$ і формула має вигляд
	\begin{equation*}
		\mu(\omega(x,y)) = \exp\left\{\int \frac{\frac{\partial M}{\partial y} - \frac{\partial N}{\partial x} }{2 x N \mp 2 y M} \diff (x^2 \pm y^2)\right\}.
	\end{equation*}

	\item Нехай $\omega(x, y) = x y$. Тоді $\partial \omega / \partial x = y$, $\partial \omega / \partial y = x$, $\diff \omega = \diff (xy)$ і формула має вигляд
	\begin{equation*}
		\mu(\omega(x,y)) = \exp\left\{\int \frac{\frac{\partial M}{\partial y} - \frac{\partial N}{\partial x} }{yN-xM} \diff (xy)\right\}.
	\end{equation*}
\end{enumerate}
