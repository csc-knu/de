Якщо ліва частина диференціального рівняння
\begin{equation*}
	M(x, y) \diff x + N(x, y) \diff y = 0,
\end{equation*}
є повним диференціалом деякої функції $u(x, y)$, тобто
\begin{equation*}
	\diff u(x, y) = M(x, y) \diff x + N(x, y) \diff y,
\end{equation*}
і, таким чином, рівняння набуває вигляду $\diff u (x, y) = 0$ то рівняння називається рівнянням в повних диференціалах. Звідси вираз
\begin{equation*}
	u(x, y) = C
\end{equation*}
є загальним інтегралом диференціального рівняння. \parvskip

Критерієм того, що рівняння є рівнянням в повних диференціалах, тобто необхідною та достатньою умовою, є виконання рівності
\begin{equation*}
	\frac{\partial M(x, y)}{\partial y} = \frac{\partial N(x, y)}{\partial x}.
\end{equation*}
 
Нехай маємо рівняння в повних диференціалах. Тоді
\begin{equation*}
	\frac{\partial u(x, y)}{\partial x} = M(x, y), \quad \frac{\partial u(x, y)}{\partial y} = N(x, y).
\end{equation*}

Звідси 
\begin{equation*}
    u(x, y) = \int M(x, y) \diff x + \phi(y),
\end{equation*}
де $\phi(y)$ --- невідома функція. Для її визначення продиференціюємо співвідношення по $y$ і прирівняємо $N(x, y)$:
\begin{equation*}
	\frac{\partial u(x, y)}{\partial y} = \frac{\partial}{\partial y} \left( \int M(x, y) \diff x \right) + \frac{\diff \phi(y)}{\diff y} = N(x, y).
\end{equation*}

Звідси
\begin{equation*}
	\phi(y) = \int \left( N(x, y) - \frac{\partial}{\partial y} \left( \int M(x, y) \diff x \right) \right) \diff y.
\end{equation*}

Остаточно, загальний інтеграл має вигляд
\begin{equation*}
	\int M(x, y) \diff x + \int \left( N(x, y) - \frac{\partial}{\partial y} \left( \int M(x, y) \diff x \right) \right) \diff y = C.
\end{equation*}

Як відомо з математичного аналізу, якщо відомий повний диференціал, то функцію $u(x, y)$ можна визначити, взявши криволінійний інтеграл по довільному контуру, що з'єднує фіксовану точку $(x_0, y_0)$ і точку із змінними координатами $(x, y)$. \parvskip

Більш зручно брати криву, що складається із двох відрізків прямих. В цьому випадку криволінійний інтеграл розпадається на два простих інтеграла
\begin{multline*}
	u(x, y) = \int_{(x_0, y_0)}^{(x,y)} M(x,y) \diff x + N(x, y) \diff y = \\
	= \int_{(x_0, y_0)}^{(x,y_0)} M(x,y) \diff x + \int_{(x,y_0)}^{(x,y)} N(x, y) \diff y = \\
	= \int_{x_0}^{x} M(\xi,y_0) \diff \xi + \int_{y_0}^{y} N(x, \eta) \diff \eta.
\end{multline*}

У цьому випадку одразу одержуємо розв'язок задачі Коші.
\begin{equation*}
	\int_{x_0}^{x} M(\xi,y_0) \diff \xi + \int_{y_0}^{y} N(x, \eta) \diff \eta = 0.
\end{equation*}
