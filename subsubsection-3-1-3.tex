\begin{definition}
	Функції $y_0(x), y_1(x), \ldots, y_n(x)$ називаються лінійно залежними на відрізку $[a,b]$ якщо існують не всі рівні нулю сталі $C_0, \ldots, C_n$ такі, що при всіх $x \in [a,b]$: 
	\begin{equation*}
		%\label{eq:3.1.18}
		C_0 \cdot y_0(x) + C_1 \cdot y_1(x) + \ldots + C_n \cdot y_n(x) = 0.
	\end{equation*}

	Якщо ж тотожність справедлива лише коли $C_0 = C_1 = \ldots = C_n = 0$, то функції $y_1(x), y_2(x), \ldots, y_n(x)$ називаються лінійно незалежними.
\end{definition}

\textbf{Приклади:}
\begin{enumerate}
	\item Функції $1, x, x^2, \ldots, x^n$ -- лінійно незалежні на будь-якому відрізку $[a,b]$, тому що вираз $C_0 + C_1 x + \ldots + C_n x^n$ є многочленом ступеню $n$ і має не більш, ніж $n$ дійсних коренів.
	\item Функції $e^{\lambda_1 x}, e^{\lambda_2 x}, \ldots, e^{\lambda_n x}$, де всі $\lambda_i$ -- дійсні різні числа -- лінійно незалежні. 
	\item Функції $1, \sin x, \cos x, \ldots, \sin nx, \cos nx$ -- лінійно незалежні.
\end{enumerate}

\begin{theorem}[необхідна умова лінійної незалежності функцій]
	Якщо функції $y_0(x), y_1(x), \ldots, y_n(x)$ -- лінійно залежні, то визначник Вронського $W[y_0, y_1, \ldots, y_n](x)$ тотожно дорівнює нулю при всіх $x \in [a,b]$:
	\begin{equation*}
		%\label{eq:3.1.19}
		W[y_0, y_1, \ldots, y_n](x) = \begin{vmatrix} y_0(x) & y_1(x) & \cdots & y_n(x) \\ y_0'(x) & y_1'(x) & \cdots & y_n'(x) \\ \vdots & \vdots & \ddots & \vdots \\ y_0^{(n)}(x) & y_1^{(n)}(x) & \cdots & y_n^{(n)}(x) \end{vmatrix} = 0.
	\end{equation*}
\end{theorem}

\begin{proof}
	Нехай $y_0(x), y_1(x), \ldots, y_n(x)$ -- лінійно залежні. Тоді існують не всі рівні нулю сталі $C_0, \ldots, C_n$ такі, що при $x \in [a,b]$ буде тотожно виконуватися
	\begin{equation*}
		%\label{eq:3.1.18}
		C_0 \cdot y_0(x) + C_1 \cdot y_1(x) + \ldots + C_n \cdot y_n(x) = 0.
	\end{equation*}
	Продиференціювавши $n$ разів, одержимо 
	\begin{equation*}
		%\label{eq:3.1.20}
		\left\{ \begin{aligned}
			C_0 \cdot y_0(x) + C_1 \cdot y_1(x) + \ldots + C_n \cdot y_n(x) &= 0, \\
			C_0 \cdot y_0'(x) + C_1 \cdot y_1'(x) + \ldots + C_n \cdot y_n(x) &= 0, \\
			\ldots \ldots \ldots \ldots \ldots \ldots \ldots \ldots \ldots \ldots \ldots \ldots \ldots & \ldots \ldots \\
			C_0 \cdot y_0^{(n)}(x) + C_1 \cdot y_1^{(n)}(x) + \ldots + C_n \cdot y_n^{(n)}(x) &= 0.
		\end{aligned} \right.
	\end{equation*}
 
	Для кожного фіксованого $x \in [a,b]$ одержимо лінійну однорідну систему алгебраїчних рівнянь, що має ненульовий розв’язок $C_0, \ldots, C_n$. А це можливо тоді і тільки тоді, коли визначник системи дорівнює нулю, тобто $W[y_0, y_1, \ldots, y_n](x) = 0$ при всіх $x \in [a,b]$.
\end{proof}

\begin{theorem}[достатня умова лінійної незалежності розв’язків]
	Якщо розв’язки лінійного однорідного рівняння $y_0(x), y_1(x), \ldots, y_n(x)$ -- лінійно незалежні, то визначник Вронського $W[y_0, y_1, \ldots, y_n](x)$ не дорівнює нулю в жодній точці $x \in [a,b]$.
\end{theorem} 

\begin{proof}
	Припустимо, від супротивного, що існує $x_0 \in [a,b]$, при якому $W[y_0, y_1, \ldots, y_n](x_0) = 0$. Оскільки визначник дорівнює нулю, то існує ненульовий розв’язок $C_0^0, C_1^0, \ldots, C_n^0$ лінійної однорідної системи алгебраїчних рівнянь \begin{equation*}
		%\label{eq:3.1.20}
		\left\{ \begin{aligned}
			C_0 \cdot y_0(x) + C_1 \cdot y_1(x) + \ldots + C_n \cdot y_n(x) &= 0, \\
			C_0 \cdot y_0'(x) + C_1 \cdot y_1'(x) + \ldots + C_n \cdot y_n(x) &= 0, \\
			\ldots \ldots \ldots \ldots \ldots \ldots \ldots \ldots \ldots \ldots \ldots \ldots \ldots & \ldots \ldots \\
			C_0 \cdot y_0^{(n)}(x) + C_1 \cdot y_1^{(n)}(x) + \ldots + C_n \cdot y_n^{(n)}(x) &= 0.
		\end{aligned} \right.
	\end{equation*}
	Розглянемо лінійну комбінацію 
	\begin{equation*}
		%\label{eq:3.1.21}
		y(x) = C_0^0 y_0(x) + C_1 y_1(x) + \ldots + C_n y_n(x)
	\end{equation*}
	з отриманими коефіцієнтами. \\

	У силу третьої властивості ця комбінація буде розв’язком. У силу вибору сталих $C_0^0, C_1^0, \ldots, C_n^0$, розв’язок буде задовольняти умовам
	\begin{equation*}
		%\label{eq:3.1.22}
		y(x_0) = y'(x_0) = \ldots = y^{(n)}(x_0) = 0.
	\end{equation*}
 
	Але цим же умовам, як неважко перевірити простою підстановкою, задовольняє і тотожний нуль, тобто $y \equiv 0$. І в силу теореми існування та єдиності ці два розв’язки співпадають, тобто 
	\begin{equation*}
		%\label{eq:3.1.23}
		y(x) = C_0^0 y_0(x) + C_1 y_1(x) + \ldots + C_n y_n(x) = 0
	\end{equation*}
	при $x \in [a,b]$, або система функцій $y_0(x), y_1(x), \ldots, y_n(x)$ лінійно залежна, що суперечить припущенню. Таким чином $W[y_0, y_1, \ldots, y_n](x_0) \ne 0$ у жодній точці $x_0 \in [a,b]$, що і було потрібно довести .
\end{proof}

На підставі попередніх двох теорем сформулюємо необхідні і достатні умови лінійної незалежності розв’язків лінійного однорідного рівняння.

\begin{theorem}
	Для того щоб розв’язки лінійного однорідного диференціального рівняння $y_0(x), y_1(x), \ldots, y_n(x)$ були лінійно незалежними, необхідно і достатньо, щоб визначник Вронського не дорівнював нулю в жодній точці $x \in [a,b]$, тобто $W[y_0, y_1, \ldots, y_n](x) \ne 0$.
\end{theorem}

\begin{theorem}
	Загальним розв’язком лінійного однорідного рівняння
	\begin{equation*}
		%\label{eq:3.1.24}
		a_0(x) \cdot y^{(n)} + a_1(x) \cdot y^{(n-1)} + \ldots + a_{n-1}(x) \cdot y' + a_n \cdot y = 0
	\end{equation*}
 	є лінійна комбінація $n$ лінійно незалежних розв’язків $y = \sum_{i = 1}^n C_i \cdot y_i(x)$.
\end{theorem}

\begin{proof}
	Оскільки $y_i(x)$, $i = 1, 2, \ldots, n$ є розв’язками, то в силу третьої властивості їхня лінійна комбінація також буде розв’язком. \\

	Покажемо, що цей розв’язок загальний, тобто вибором сталих $C_1, \ldots, C_n$ можна розв’язати довільну задачу Коші
	\begin{equation*}
		%\label{eq:3.1.25}
		y(x_0) = y_0, \quad y'(x_0) = y_0', \quad \ldots, \quad y^{(n - 1)}(x_0) = y_0^{(n - 1)}.
	\end{equation*}

	Дійсно, оскільки система розв’язків лінійно незалежна, то визначник Вронського відмінний від нуля й алгебраїчна система неоднорідних рівнянь
	\begin{equation*}
		%\label{eq:3.1.26}
		\left\{ \begin{aligned}
			C_1 \cdot y_1(x_0) + C_2 \cdot y_2(x_0) + \ldots + C_n \cdot y_n(x_0) &= y_0, \\
			C_1 \cdot y_1'(x_0) + C_2 \cdot y_2'(x_0) + \ldots + C_n \cdot y_n(x_0) &= y_0', \\
			\ldots \ldots \ldots \ldots \ldots \ldots \ldots \ldots \ldots \ldots \ldots \ldots \ldots \ldots & \ldots \ldots \\
			C_1 \cdot y_1^{(n - 1)}(x_0) + C_2 \cdot y_2^{(n - 1)}(x_0) + \ldots + C_n \cdot y_n^{(n)}(x_0) &= y_0^{(n - 1)},
		\end{aligned} \right.
	\end{equation*}
	має єдиний розв’язок $C_1^0, C_2^0, \ldots, C_n^0$. І лінійна комбінація $y = \sum_{i = 1}^n C_i^0 \cdot y_i(x)$ є розв’язком, причому, як видно із системи алгебраїчних рівнянь, буде задовольняти довільно обраним умовам Коші.
\end{proof}

Зауважимо, що максимальне число лінійно незалежних розв’язків дорівнює порядку рівняння. Це випливає з попередньої теореми, тому що будь-який розв’язок виражається через лінійну комбінацію $n$ лінійно незалежних розв’язків.

\begin{definition}
	Будь-які $n$ лінійно незалежних розв’язків лінійного однорідного рівняння $n$-го порядку називаються фундаментальною системою розв’язків.
\end{definition}